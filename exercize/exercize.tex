\documentclass{article}

\title{Laramon - A Laravel Practice Project}
\author{Wesley van Schaijk}
\date{2024-03-11}

\begin{document}

\maketitle

\paragraph{}
\textbf{Before starting with this exercize, 
make sure that sqlite is enabled in your php.ini file.}

\paragraph{}
Laramon is a website dedicated to building a Pokémon team.
You have been tasked to add the latest features to this website.
The current state of the project is accessible through the starter code.
Finish the website by completing the following tasks:

\paragraph{Base Exercizes - Pokémon CRUD}
During the base exercize, you are required to make the Pokémon appear on the screen.
The company that started the website have not created any views, this is all default.
The only thing you have is some data files.
Make sure the relations are set up correctly, and all information of the Pokémon are displayed.
It's up to you to decide what information is displayed in the index, and what in show.
You do not need CRUD functionality for Types, only Pokémon and Pokémon Types.

\begin{enumerate}
	\item Finish the migrations for Pokémon types and Pokémon. 
          An example can be found in Types. You do not have to deal with nullable values.
          It does include relations.
    \item Add the fillable components to the models for Pokémon types and Pokémon.
    \item Finish the seeders for Pokémon types and Pokémon. Use the TypeSeeder as example.
    \item Make sure that all seeders run.
    \item Create the CRUD functionality in the controllers, create views accordingly.
    \item Make the routing for the Pokémon.
\end{enumerate}

\paragraph{Extra Exercizes}

\begin{enumerate}
    \item Make a model, migration, and resource controller for Teams.
    \item Create CRUD functionality for the user to select and save teams of six Pokémon.
    \item Make view accordingly.
    \item Make images appear through of Pokémon, instead of just raw data.
          Hint, you can steal images from pokemondb.net.
          https://img.pokemondb.net/artwork/avif/bulbasaur.avif
\end{enumerate}

\end{document}